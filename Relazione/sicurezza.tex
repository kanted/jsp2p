\documentclass[a4paper,11pt,twoside,openright]{book} % A4, 11 punti, fronte-retro, libro
\usepackage[italian]{babel} % Adatta LaTeX alle convenzioni tipografiche italiane
\usepackage[utf8]{inputenc} % Consente l'uso dei caratteri accentati italiani
\usepackage{graphicx} % Per le immagini  
\usepackage {fancyhdr} % Per l' ambiente abstract
\usepackage{indentfirst} % Indentazione all' inizio dei capitoli
\usepackage{url} % Formattazione URL
\usepackage{amsmath} % Vari simboli matematici
\usepackage{amssymb} %Per l' "uguale per definizione"
\usepackage{amsthm} % Per le dimostrazioni
\usepackage{amsfonts} % Per font matematici
\usepackage{mathrsfs} % Per i font corsivi belli, tipo la N delle gaussiane ecc...
\usepackage{listings} % Per il codice
\usepackage{easybmat} % Per le matrici a blocchi
\usepackage{mathtools}
\usepackage{color}
\newenvironment{mylisting} % Supporto al codice sorgente
	{\begin{list}{}{\setlength{\leftmargin}{1em}}\item\scriptsize\bfseries} %
	{\end{list}}%
\newenvironment{mytinylisting} %
	{\begin{list}{}{\setlength{\leftmargin}{1em}}\item\tiny\bfseries} %
	{\end{list}} %
\newcommand{\fncyblank}{\fancyhf{}} % 
\newenvironment{abstract} %
\frenchspacing % Non aumentare la spaziatura tra periodi

\newlength{\symlen}
\newlength{\barlen}

\newcommand{\overstrike}[2]{\mbox{\settowidth{\symlen}{$#1$}%
        $#1$\hspace{-\symlen}$#2$}}

\settowidth{\barlen}{$|$}
\newcommand{\vbarred}[1]{\,\overstrike{\hspace{-0.25 \barlen}|}{#1}\,}

\newcommand{\sees}[2]{#1 \triangleleft #2}
\newcommand{\said}[2]{#1 \vbarred{\sim} #2}
\newcommand{\believes}[2]{#1 \vbarred{\equiv} #2}
\newcommand{\fresh}[1]{\sharp\left(#1\right)}
\newcommand{\sharedkey}[3]{#1\xleftrightarrow{#3}#2}
\newcommand{\sharedsecret}[3]{#1\overset{#3}{\leftrightharpoons} #2}
\newcommand{\publickeyowner}[2]{\stackrel{#1}{\mapsto}#2}
\newcommand{\publickey}[1]{{K_{#1}}^+}
\newcommand{\privatekey}[1]{{K_{#1}}^-}
\newcommand{\encrypt}[2]{\left\{#1\right\}_{#2}}
\newcommand{\decrypt}[2]{\left\{#1\right\}^{-1}_{#2}}
\newcommand{\combine}[2]{\left\langle #1\right\rangle_{#2}}
\newcommand{\jurisdiction}[2]{#1 \Rightarrow #2}

\begin{document}
	\frontmatter
		%%%%%%%%%%%%%%%%%%%%%%%%%%%%%%%%%%%%%%%%%%%%%%%%%%%%%%%%%%%
% Frontespizio

% vspace serve ad aggiungere extra spazio verticale
% em sta ad indicare la grandezza della lettera M maiuscola

% Large indica una dimensione del font di 14.4 pt
% large indica una dimensione del font di 12 pt
% normalsize indica una dimensione del font di 10 pt

% vfill inserisce sufficiente spazio binaco verticalmente per fare in modo che il
% sopra e il sotto del testo siano allieneati col margine superiore e inferiore

\begin{titlepage}
 \begin{center}
     \vspace{1em}
     {\Large \textsc{Università degli studi di Pisa}}\\
     \vspace{1em}
     {\Large \textsc{Facoltà di Ingegneria}}\\
     \vspace{2em}
     {\normalsize Laurea Magistrale in}\\
     \vspace{1em}
     {\Large \textsc{Ingegneria Informatica}}\\
     \vspace{2em}
     {\Large \textsc{Sicurezza nei sistemi informatici}}\\
     \vspace{2em}
     {\LARGE \textbf{Progetto di una applicazione peer-to-peer}}\\
     \vspace{1em}
     {\LARGE \textbf{con comunicazione sicura}}
 \end{center}

\vskip 2cm
  \begin{center}
      Implementazione in Java
      \vskip 1cm
      a cura di\\
      \emph{Sacco Cosimo} e \emph{Silvestri Davide}
  \end{center}

\vskip 1cm
\begin{center}
{\normalsize Anno Accademico 2010/2011}
\end{center}
\end{titlepage}
\clearpage{\pagestyle{empty}\cleardoublepage}
%\clearpage{\pagestyle{empty}\cleardoublepage}

		\tableofcontents
\clearpage{\pagestyle{empty}\cleardoublepage}
%\listoffigures
%\clearpage{\pagestyle{empty}\cleardoublepage}
%\listoftables
%\clearpage{\pagestyle{empty}\cleardoublepage}

	\mainmatter
		%\input{introduzione.tex}
		\chapter{Protocollo}
\label{chap:protocollo}
	Il sistema è stato pensato per garantire confidenzialità nella comunicazione in un sistema 
	peer-to-peer in cui ciascun membro dispone di una coppia di chiavi pubblica e privata 
	opportunamente certificata. In particolare, la comunicazione tra i peer è preceduta dall' 
	esecuzione del protocollo per stabilire una chiave di sessione. Tale protocollo garantisce 
	\emph{key authentication} (ovvero, ogni peer crede che la chiave stabilita sia la chiave
	di sessione), \emph{key confirmation} (ovvero, ogni peer crede che la controparte sia
	convinta sulla key authentication relativamente alla chiave di sessione) e infine
	\emph{key freshness} (si garantisce che entrambe le parti siano convinte della freschezza
	della chiave di sessione generata).
	Viene illustrato, qui di seguito, il protocollo:
	\[
		\begin{aligned}
			M1:\ & A \rightarrow B & & C_A\\
			M2:\ & A \leftarrow B & & C_B\\
			M3:\ & A \rightarrow B & & \encrypt{n_A}{\publickey{B}}\\
			M4:\ & A \leftarrow B & & \encrypt{n_A,\ n_B}{\publickey{A}}\\
			M5:\ & A \rightarrow B & & \encrypt{n_B}{\publickey{B}}\\
		\end{aligned}
	\]
	Significato dei messaggi:
	\begin{description}
		\item[M1]: $A$ invia a $B$ il proprio certificato $C_A$;
		\item[M2]: avendo verificato la validità del certificato di $A$, $B$ invia a $A$ il proprio certificato $C_B$;
		\item[M3]: avendo verificato la validità del certificato di $B$, $A$ genera un nonce random $n_A$ e lo invia
		cifrandolo con la chiave pubblica di $B$. In questo modo, solamente B potrà leggere $M3$;
		\item[M4]: $B$ decifra il nonce con la sua chiave privata, a sua volta crea un nonce random $n_B$ ed invia
		ad $A$ $n_A$ ed $n_B$ cifrando il messaggio con la chiave pubblica di $A$. In questo modo, solo $A$ potrà 
		leggere $M4$;
		\item[M5]: $A$ decifra il nonce $n_A$ e controlla che corrisponda a quello che ha inviato. In caso affermativo,
		reinvia a $B$ il nonce $n_B$, dopo averlo decifrato con la sua chiave privata. Infine, $A$ genera la chiave di sessione.
		$B$ verifica che il nonce $n_B$ ricevuto corrisponda a quello precedente inviato. In tal caso, procede a generare la chiave di sessione.
	\end{description}

	La chiave di sessione viene generata a partire dai nonce $n_A$ ed $n_B$. In questo modo entrambe le parti partecipano
	alla creazione della chiave, quindi non è necessario che i peer facciano assunzioni sulla capacità
	della controparte di generare nonce realmente freschi.
	L'algoritmo di generazione della chiave di sessione è il seguente:
	\[
		\begin{cases}
			y = h ( n_A || n_B )\\
			K_{AB} =  \overleftarrow{T_{128bit}}(y)
		\end{cases}
	\]
	dove per $\overleftarrow{T_{128bit}}(y)$ si intende il troncamento di $y$ ai suoi $128\ bit$ più significativi.
\clearpage{\pagestyle{empty}\cleardoublepage}

		

\chapter{Analisi del protocollo di scambio chiavi}
\section{\emph{Beliefs} da ottenere}
\label{sec:beliefs}
	Procediamo ad analizzare il protocollo esposto nel capitolo \ref{chap:protocollo}.
	Si vuole provare che il protocollo produce, in ciascuna delle parti, i seguenti \emph{beliefs}:
	\begin{center}
		\begin{tabular}{| c | c | c |}
			\hline
			\ & {\bf A} & {\bf B} \\
			\hline
			{\bf key authentication} & $\believes{A}{\sharedkey{A}{B}{K}}$ & $\believes{B}{\sharedkey{A}{B}{K}}$\\
			\hline
			{\bf key confirmation} & $\believes{A}{\believes{B}{\sharedkey{A}{B}{K}}}$ &
			                         $\believes{B}{\believes{A}{\sharedkey{A}{B}{K}}}$\\
			\hline
			{\bf key freshness} & $\believes{A}{\fresh{\sharedkey{A}{B}{K}}}$ & $\believes{B}{\fresh{\sharedkey{A}{B}{K}}}$\\
			\hline
		\end{tabular}
	\end{center}
\section{Protocollo idealizzato}
	Viene riportato, qui di seguito, il \emph{protocollo idealizzato} relativo
	al protocollo di scambio delle chiavi esposto nel capitolo \ref{chap:protocollo}.
	\[
		\begin{aligned}
			M1:\ & A \rightarrow B & & \encrypt{\publickeyowner{\publickey{A}}{A}, \ L_A}{\privatekey{T}}\\
			M2:\ & A \leftarrow B & & \encrypt{\publickeyowner{\publickey{B}}{B}, \ L_B}{\privatekey{T}}\\
			M3:\ & A \rightarrow B & & \encrypt{n_A,\ \sharedsecret{A}{B}{n_A}}{\publickey{B}}\\
			M4:\ & A \leftarrow B & & \encrypt{n_A,\ n_B,\ \sharedkey{A}{B}{\combine{n_A}{n_B}}}{\publickey{A}}\\
			M5:\ & A \rightarrow B & & \encrypt{n_B,\ \sharedkey{A}{B}{\combine{n_A}{n_B}}}{\publickey{B}}\\
		\end{aligned}
	\]
\section{Ipotesi}
	\label{sec:ipotesi}
	Vengono esplicitate, qui di seguito, le ipotesi sotto le quali il protocollo viene eseguito.
	\begin{center}
		\begin{tabular}{| c | c | c |}
			\hline
			\ & {\bf A} & {\bf B} \\
			\hline
			{\bf public keys} & $\believes{A}{\publickeyowner{\publickey{A}}{A}}$ & $\believes{B}{\publickeyowner{\publickey{B}}{B}}$\\
			\hline
			{\bf third party} & $\believes{A}{\publickeyowner{\publickey{T}}{T}}$ & $\believes{B}{\publickeyowner{\publickey{T}}{T}}$\\
			                \ & $\believes{A}{\jurisdiction{T}{\publickeyowner{\publickey{X}}{X}}}$ %
			                  & $\believes{B}{\jurisdiction{T}{\publickeyowner{\publickey{X}}{X}}}$\\
			\hline
			{\bf freshness} &  $\believes{A}{\fresh{n_A}}$ & $\believes{B}{\fresh{n_B}}$\\
			              \ &  $\believes{A}{\fresh{L_B}}$ & $\believes{B}{\fresh{L_A}}$\\
			\hline
		\end{tabular}
	\end{center}
\section{Analisi dei \emph{beliefs}}
	Procediamo, ora, con l' analisi dei singoli messaggi. Partendo dalle ipotesi esposte nella sezione
	\ref{sec:ipotesi} e applicando le \emph{regole di inferenza} della logica BAN, ciascuna parte può ampliare
	l' insieme dei propri \emph{beliefs}.
	Se, tra i beliefs finali, compaiono quelli elencati nella sezione \ref{sec:beliefs},
	allora possiamo affermare che il protocollo esposto è corretto.
	\subsection{Messaggio $M1$}
		Messaggio $M1$:
		\[
			\begin{aligned}
				M1:\ & A \rightarrow B & & \encrypt{\publickeyowner{\publickey{A}}{A}, \ L_A}{\privatekey{T}}\\
			\end{aligned}
		\]
		per la \emph{meaning rule}
		\[
			\frac{\believes{B}{\publickeyowner{\publickey{T}}{T}},\ \sees{B}{\encrypt{\publickeyowner{\publickey{A}}{A},\ L_A}{\privatekey{T}}}}
			{\believes{B}{\said{T}{\left(\publickeyowner{\publickey{A}}{A},\ L_A\right)}}}
		\]
		e poiché
		\[
			\frac{\believes{B}{\fresh{L_A}}}{\believes{B}{\fresh{\publickeyowner{\publickey{A}}{A}, \ L_A}}}
		\]
		allora, per la \emph{nonce verification rule}
		\[
			\frac{\believes{B}{\said{T}{\left(\publickeyowner{\publickey{A}}{A},\ L_A\right)}},\ \believes{B}{\fresh{\publickeyowner{\publickey{A}}{A},\ L_A}}}
			{\believes{B}{\believes{T}{\left(\publickeyowner{\publickey{A}}{A},\ L_A\right)}}}
		\]
		e, in particolare,
		\[
			\believes{B}{\believes{T}{\publickeyowner{\publickey{A}}{A}}}
		\]
		infine, per la \emph{jurisdiction rule}
		\[
			\frac{\believes{B}{\believes{T}{\publickeyowner{\publickey{A}}{A}}},\ \believes{B}{\jurisdiction{T}{\publickeyowner{\publickey{A}}{A}}}}
			{\believes{B}{\publickeyowner{\publickey{A}}{A}}}
		\]
	\subsection{Messaggio $M2$}
		In maniera del tutto analoga a quanto visto per il messaggio $M1$, il \emph{belief} ottenuto da $A$ dopo aver ricevuto
		il messaggio $M2$ è
		\[
			\believes{A}{\publickeyowner{\publickey{B}}{B}}
		\]
	\subsection{Messaggio $M3$}
	Messaggio $M3$:
		\[
			\begin{aligned}
				M3:\ & A \rightarrow B & & \encrypt{n_A,\ \sharedsecret{A}{B}{n_A}}{\publickey{B}}\\
			\end{aligned}
		\]
		L' applicazione delle regole di inferenza non porta, su $B$, alla realizzazione di alcun nuovo belief.
		Tuttavia, poiché l' unica entità in grado di leggere il nonce $n_A$ è $B$\footnote{$B$, infatti, è l' unica
		entità a possedere la chiave $\privatekey{B}$ necessaria per decifrare i messaggi cifrati con $\publickey{B}$.},
		$A$ può ritenere che
		\[
			\believes{A}{\sharedsecret{A}{B}{n_A}}
		\]
	\subsection{Messaggio $M4$}
	Messaggio $M4$:
		\[
			\begin{aligned}
				M4:\ & A \leftarrow B & & \encrypt{n_A,\ n_B,\ \sharedkey{A}{B}{\combine{n_A}{n_B}}}{\publickey{A}}\\
			\end{aligned}
		\]
		L' unica entità in grado di leggere il messaggio $M4$ è $A$\footnote{$A$, infatti, è l' unica
		entità a possedere la chiave $\privatekey{A}$ necessaria per decifrare i messaggi cifrati con $\publickey{A}$.}.
		Pertanto, $B$ può ritenere che
		\[
			\believes{B}{\sharedkey{A}{B}{\combine{n_A}{n_B}}}\ \ \ \ \ \ \text{\emph{B ottiene key authentication}}
		\]
		inoltre,
		\[
			\frac{\believes{B}{\fresh{n_B}}}
			{\believes{B}{\fresh{\sharedkey{A}{B}{\combine{n_A}{n_B}}}}}\ \ \ \ \ \ \text{\emph{B ottiene key freshness}}
		\]
		per quanto riguarda $A$, invece, otteniamo
		\[
			\frac{\believes{A}{\fresh{n_A}}}
			{\believes{A}{\fresh{n_A,\ n_B,\ \sharedkey{A}{B}{\combine{n_A}{n_B}}}}}
		\]
		e poiché, per la \emph{meaning rule}
		\[
			\frac{\believes{A}{\sharedsecret{A}{B}{n_A}},\ \sees{A}{\left(n_A,\ n_B,\ \sharedkey{A}{B}{\combine{n_A}{n_B}}\right)}}
			{\believes{A}{\said{B}{\left(n_A,\ n_B,\ \sharedkey{A}{B}{\combine{n_A}{n_B}}\right)}}}
		\]
		allora, per la \emph{nonce verification rule}
		\[
			\frac{\believes{A}{\said{B}{\left(n_A,\ n_B,\ \sharedkey{A}{B}{\combine{n_A}{n_B}}\right)}},
			\ \believes{A}{\fresh{n_A,\ n_B,\ \sharedkey{A}{B}{\combine{n_A}{n_B}}}}}
			{\believes{A}{\believes{B}{\left(n_A,\ n_B,\ \sharedkey{A}{B}{\combine{n_A}{n_B}}\right)}}}
		\]
		e in particolare
		\[
			\believes{A}{\believes{B}{\sharedkey{A}{B}{\combine{n_A}{n_B}}}}\ \ \ \ \ \ \text{\emph{A ottiene key confirmation}}
		\]
	\subsection{Messaggio $M5$}
	Messaggio $M5$:
		\[
			\begin{aligned}
				M5:\ & A \rightarrow B & & \encrypt{n_B,\ \sharedkey{A}{B}{\combine{n_A}{n_B}}}{\publickey{B}}\\
			\end{aligned}
		\]
		L' unica entità in grado di leggere il messaggio $M5$ è $B$.
		Pertanto, $A$ può ritenere che
		\[
			\believes{A}{\sharedkey{A}{B}{\combine{n_A}{n_B}}}\ \ \ \ \ \ \text{\emph{A ottiene key authentication}}
		\]
		inoltre,
		\[
			\frac{\believes{A}{\fresh{n_A}}}
			{\believes{A}{\fresh{\sharedkey{A}{B}{\combine{n_A}{n_B}}}}}\ \ \ \ \ \ \text{\emph{A ottiene key freshness}}
		\]
		per quanto riguarda $B$, invece, otteniamo
		\[
			\frac{\believes{B}{\fresh{n_B}}}
			{\believes{B}{\fresh{n_B,\ \sharedkey{A}{B}{\combine{n_A}{n_B}}}}}
		\]
		e poiché, per la \emph{meaning rule}
		\[
			\frac{\believes{B}{\sharedkey{A}{B}{\combine{n_A}{n_B}}},\ \sees{B}{\left(n_B,\ \sharedkey{A}{B}{\combine{n_A}{n_B}}\right)}}
			{\believes{B}{\said{A}{\left(n_B,\ \sharedkey{A}{B}{\combine{n_A}{n_B}}\right)}}}
		\]
		allora, per la \emph{nonce verification rule}
		\[
			\frac{\believes{B}{\said{A}{\left(n_B,\ \sharedkey{A}{B}{\combine{n_A}{n_B}}\right)}},
			\ \believes{B}{\fresh{n_B,\ \sharedkey{A}{B}{\combine{n_A}{n_B}}}}}
			{\believes{B}{\believes{A}{\left(n_B,\ \sharedkey{A}{B}{\combine{n_A}{n_B}}\right)}}}
		\]
		e in particolare
		\[
			\believes{B}{\believes{A}{\sharedkey{A}{B}{\combine{n_A}{n_B}}}}\ \ \ \ \ \ \text{\emph{B ottiene key confirmation}}
		\]
\clearpage{\pagestyle{empty}\cleardoublepage}

		%\input{progetto.tex}
		%\chapter{Installazione}
Installazione ed esecuzione:
\begin{itemize}
	\item Installare il provider Bouncy Castle secondo la procedura seguente:
	\begin{itemize}
		\item scaricare il package BC dal sito di riferimento
		\begin{center}
			\url{http://www.bouncycastle.org/latest_releases.html});
		\end{center}
		\item copiare il package nella directory \verb1<JAVA_HOME>/jre/lib/ext1;
		\item modificare il file \verb1<JAVA_HOME>/jre/lib/security/java.security1
		aggiungendo, come ultimo elemento della lista che ha il formato
		\begin{center}
			\verb1security.provider.1\textit{n}\verb1=provider1
		\end{center}
		la riga
		\begin{center}
			\verb1security.provider.1\textit{n+1}\verb1=org.bouncycastle.jce.provider.BouncyCastleProvider1
		\end{center}
	\end{itemize}
	\item scompattare l'archivio \verb1progetto_sicurezza_sacco_silvestri.jar1
	\item copiare il contenuto della cartella \verb1file di configurazione1 nella directory
	\verb1progetto_sicurezza_sacco_silvestri1 appena creata
	\item eseguire Trent
	\item eseguire Peer
\end{itemize}
\clearpage{\pagestyle{empty}\cleardoublepage}

\end{document}
\clearpage{\pagestyle{empty}\cleardoublepage}
