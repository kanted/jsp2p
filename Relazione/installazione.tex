{\rtf1\ansi\ansicpg1252\cocoartf1038\cocoasubrtf360
{\fonttbl\f0\fswiss\fcharset0 Helvetica;\f1\froman\fcharset0 Times-Roman;}
{\colortbl;\red255\green255\blue255;\red83\green83\blue83;}
\paperw11900\paperh16840\margl1440\margr1440\vieww9000\viewh8400\viewkind0
\pard\tx566\tx1133\tx1700\tx2267\tx2834\tx3401\tx3968\tx4535\tx5102\tx5669\tx6236\tx6803\ql\qnatural\pardirnatural

\f0\fs24 \cf0 Installazione ed esecuzione:\
 - Installare il provider Bouncy Castle secondo la procedura seguente:
\f1\fs23\fsmilli11996 	1)scaricare il package BC dal sito di riferimento (\cf2 http://www.bouncycastle.org/latest_releases.html\cf0 )\
\pard\tx560\tx1120\tx1680\tx2240\tx2800\tx3360\tx3920\tx4480\tx5040\tx5600\tx6160\tx6720\ql\qnatural\pardirnatural

\i \cf0           2)
\i0 copiare il package nella directory 
\i <JAVA_HOME>/jre/lib/ext 
\i0 \
          3)modificare il file 
\i <JAVA_HOME>/jre/lib/security/j
\i0 ava.security aggiungendo, come ultimo elemento della lista che ha il formato "security.provider.
\i n
\i0 =provider" la riga: "security.provider
\i .n+1
\i0 =org.bouncycastle.jce.provider.BouncyCastleProvider"\
\pard\tx566\tx1133\tx1700\tx2267\tx2834\tx3401\tx3968\tx4535\tx5102\tx5669\tx6236\tx6803\ql\qnatural\pardirnatural

\f0\fs24 \cf0 - Scompattare l'archivio progetto_sicurezza_sacco_silvestri.jar\
- Eseguire Trent.class\
- Eseguire Peer.class\
}