\chapter{Installazione}
Installazione ed esecuzione:
\begin{itemize}
	\item Installare il provider Bouncy Castle secondo la procedura seguente:
	\begin{itemize}
		\item scaricare il package BC dal sito di riferimento
		\begin{center}
			\url{http://www.bouncycastle.org/latest_releases.html});
		\end{center}
		\item copiare il package nella directory \verb1<JAVA_HOME>/jre/lib/ext1;
		\item modificare il file \verb1<JAVA_HOME>/jre/lib/security/java.security1
		aggiungendo, come ultimo elemento della lista che ha il formato
		\begin{center}
			\verb1security.provider.1\textit{n}\verb1=provider1
		\end{center}
		la riga
		\begin{center}
			\verb1security.provider.1\textit{n+1}\verb1=org.bouncycastle.jce.provider.BouncyCastleProvider1
		\end{center}
	\end{itemize}
	\item scompattare l'archivio \verb1progetto_sicurezza_sacco_silvestri.jar1
	\item copiare il contenuto della cartella \verb1file di configurazione1 nella directory
	\verb1progetto_sicurezza_sacco_silvestri1 appena creata
	\item eseguire Trent
	\item eseguire Peer
\end{itemize}
\clearpage{\pagestyle{empty}\cleardoublepage}
